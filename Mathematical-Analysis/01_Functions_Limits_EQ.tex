%% *****************************************************************************
%% Еквівалентні нескінченно малі функції для обчислення границь
%% *****************************************************************************

\begin{center}\large{\cyr{\textbf{Еквівалентні нескінченно малі функції для обчислення границь}}}\end{center}

Швидким способом знаходження границь функцій, що мають особливості виду нуль на нуль $\big[\frac{0}{0}\big]$ є застосування еквівалентних нескінченно малих функцій. Вони вкрай необхідні, якщо потрібно знаходити границі без застосування правила Лопіталя. Еквівалентності полягають в заміні функції її розкладом в ряд Маклорена. Як правило, при обчисленні границь використовують не більше двох членів розкладу.

\begin{quote}
Дві нескінченномалі функції $\alpha(x)$ і $\beta(x)$ називають еквівалентними якщо різниця їх відношення дорівнює 1

\begin{displaymath}
  \lim_{x \to 0}\frac{\alpha(x)}{\beta(x)} = 1 \qquad \text{з чого випливає} \qquad \alpha(x) \sim \beta(x)
\end{displaymath}

\textbf{Теорема}: Якщо нескінченно мала функція \mbox{$\alpha(x)$ \sim $\alpha_i(x)$} при $x\to{a}$ і \mbox{$\beta(x)$ \sim $\beta_i(x)$} при $x\to{a}$  то їх можн азамінити на еквівалентні:

\begin{displaymath}
  \lim_{x \to a}\frac{\alpha(x)}{\beta(x)} = \lim_{x \to a}\frac{\alpha_1(x)}{\beta_1(x)}
\end{displaymath}

\end{quote}

Для зручності наведемо еквівалентності основних функцій при прямуванні змінної до нуля $a \to 0 $

\begin{enumerate}
  \item $\sin(a) \sim a, a \to 0$
  \item $\arcsin(a) \sim a, a \to 0$
  \item $\cos(a) \sim 1-\frac{a^2}{2}, a \to 0$
  \item $\tan({a}) \sim a, a \to 0 $
  \item $\arctan({a}) \sim a, a \to 0 $
  \item $e^a-1 \sim a \to 0 $
  \item $a^a-1 \sim a\ln{a},  a \to 0 $
  \item $\ln(1+a) \sim a, a \to 0$
  \item $(1+a)^k \sim ka, a \to 0 $
\end{enumerate}
