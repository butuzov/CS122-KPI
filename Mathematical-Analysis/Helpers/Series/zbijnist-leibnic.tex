\M{\large{\textbf{Ознака збіжності Лейбніца}}}

\M{\normalsize{\textbf{використовується ЛИШЕ ДЛЯ ЗНАКОЗМІННИХ рядів}}}

Знакозмінний ряд сходитьс якщо абсолютні велечини його членів спадають. а загальний член ряду прямує до нуля.
$$
\begin{array}{lll}
    1) & u_1 > u_2 > u_3 > \ldots > u_n > \ldots & {\text{починаючи з деякого номеру (n > N)}}\\
    \\
    2) & \Lim{n \to \infty } u_n=0&
\end{array}
$$

\M{\textbf{Не відповідність ряду першій ознаці не є достаньою ознакою того що ряд не сходиться.}}

$$
\begin{array}{lll}
  1 & \sum_{n=1}^\infty \Big|a_n\Big| {\text{ \textbf{збігається} }}
    & \Rightarrow \sum_{n=1}^\infty a_n {\text{ \textbf{зігається абсолютно} }}
    \\
  2 &  \sum_{n=1}^\infty \Big|a_n\Big| {\text{ \textbf{розбіжний, АЛЕ Ознаки Лейбніца - ВИКОНУЮТЬСЯ } }}
    & \Rightarrow \sum_{n=1}^\infty a_n {\text{ \textbf{збіжний умовно} }}
\end{array}
$$


\M{\textbf{Дослідження знакозмінних рядів здійснюється за алгоритмом:}}
\begin{itemize}
    \item якщо $\Lim{n \to \infty} \big| u_n \big| \neq 0$, то ряд розбіжний за необхідною ознакою, інакше до пункта 2.
    \item якщо ряд $\Sum{n=1}{\infty} \big| u_n \big|$ збіжний то данний ряд збігається абсолютно, інакше пунк 3
    \item якщо ряд  $\Sum{n=1}{\infty}  u_n  $ є рядом лейбніца то він збігається умовно.
  \end{itemize}
