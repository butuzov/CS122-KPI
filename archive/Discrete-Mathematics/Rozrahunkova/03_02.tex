\begin{center}\large{\cyr{\textbf{3.2  Вибір ненумерованих об'єктів}}}\end{center}

Букет квітів може містити хразантеми, троянди, ромашки та півонії. Все квіти вважаються ненумерованими, тобто квіти одного виду між собою не розрізняються. Порядок квітів в букеті не має значення. Скільки можна скласти букетиім із $n$ квітів, якщо кожен букет повинен містити не менш ніж $k_1$ і не менш ніж $n_1$ хризантем, не менш ніж $k_2$ і не менш ніж $n_2$ троянд, не менш ніж $k_3$ і не менш ніж $n_3$ ромашок, не менш ніж $k_4$ і не менш ніж $n_4$ півоній?

$$
  \begin{array}{ lcr  }
    n &=& 10 \\
  \end{array}
  \begin{array}{ lcr }
    k_1 &=& 3 \\
    k_2 &=& 1 \\
    k_3 &=& 1 \\
    k_4 &=& 1 \\
  \end{array}
  \begin{array}{ lcr  }
    n_1 &=& 5 \\
    n_2 &=& 3 \\
    n_3 &=& 4 \\
    n_4 &=& 5 \\
  \end{array}
$$
