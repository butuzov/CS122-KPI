\section{Оболонка - tsch}

  \subsection{Екскурс в історію}
  tcsh або TENEX C Shell це Unix оболонка створена Кеном Грігом. Вона бере ім'я від операційної системи TENEX (TEN EXtended) що була розроблена на мові програмування LIST для мейнфрейму DEC PDP-10. Якщо для UNIX використовувались короткі команди наприклад \texttt{ls} (для виводу каталога), то для TENEX аналогом такої команди була \texttt{DIRECTORY (OF FILES)} де власне командою було \texttt{DIRECTORY}, а \texttt{(OF FILES)} було "шумом" або додатковою опцією ("щоб було зрозуміліше"). Ясно що такі команди було вводити трошка складніше і TENNEX мав встроєну систему автодоповнення, опції що багато UNIX систем не мали.

  Робота над tcsh почалась у версні 1975 і закінчилась фінально злиттям проекту з csh (C shell) у вересні 1983 року, через місяць першоджерельні файли були надіслані до  usernet'івської групи net.sources.

  tcsh є оболонкою по замовчуванню в відкритих операційних системах FreeBSD та також пропрієтарних, наприклад UNIX подібних системах компанії IBM - OS/390 і z/OS, та FreeBSD похідхних NextComputer та Apple (перші версії OS X використовували tcsh як основну оболонку, але починаючи з версії 10.3 оболонкою по замовчуванню є bash).

  \subsection{Основні функції}
  Як і у інших оболонок основною функцією tсsh є:
  \begin{itemize}
    \item Запуск команд або утиліт що запитує система.
    \item Написання сценаріїв оболонки використовуючи власну мову програмування.
    \item Запуск сценаріїв оболонки та програм інтерактивно або у фоні.
    \end{itemize}

  Основними можливостями tcsh в порівнянні з іншими оболонками є:
  \begin{itemize}
    \item Історія вже запущених команд
    \item Редагування командного рядка з підтримкою стилей від \texttt{vi} або \texttt{emacs}
    \item розширений механізм нафігації по каталогах
    \item Авто-доповнення як файлів так і імен змінних що користувач може друкувати в командному рядку.
    \item Аліаси до аргументів що можуть бути перенаправлені до відповідних команд.
    \item Управління задачами
    \item команда \texttt{where} (аналогічна \texttt{which}) що показувала усі локейшени шуканого обєкту.
  \end{itemize}

  \subsection{Спеціальні можливості}
  Саме нявність автодоповнення що дозволяв вводити довгі рядки команд забезпечила популярність tcsh на довгі роки, і саме до цієї особливості зводиться використання tcsh більшостю користувачів. Слід також зосередити увагу на відмінностях в роботі в порівнянні з найпоширенішою на сьогодні оболонкою bash.

  \begin{itemize}
    \item \texttt{alias}/\texttt{unalias} працють по іншому в \texttt{tcsh} ніж в \texttt{bash}, замість оператора еквівалентності (=) використовується табуляція або пробіли
    \item Не зручний варіант написання сценарії: що виражається в неможливості використання функцій декларований напряму в файлі сценарію.
    \item вбудована команда \texttt{history} що виводить останні команди (і працює трошки відмінно від аналогічної команди у bash).
    \item history expansion - схоже за схемою на bash аналог, але дозволяє більш гнучко визивати команди х історії наприклад \texttt{!123} команди в списку історії номер 123 (123 з кінця), а \texttt{!?cat?} виклче останнію команду що містиnm "cat". Слід не оминути і модифікатори, що дозволяють змінювати параметри розширення історій в процесі виконання.
    \item \texttt{set} що дозволялал змінювати певні змінні, такі як \texttt{history} (кількість записів історії що ви виведемо) і \texttt{savehist} (кількість записів у файлі \texttt{~.history}) та інших... (а такою команди \texttt{unset} що обнуляла данні значення)
    \item Спеціальні псевдоніми - виконання команд при певних умовах (наприклад автоматичне виконання команд що є псевонімами до cwdcmd перед тим як робочий каталог буде змінено)
    \item Управління задачами - працює аналогічно окрім моменту коли tcsh виводить усі PID що належать виконуваному процесу.
    \item перенаправлення stderr працює інакше - через \texttt{>\&}
    \item Автозавершення імен файлів (просто написавши початок імени файлу/папки і натиснувши \texttt{TAB}).
    \item Автозавершення команд (просто написавши початок команди і натиснувши \texttt{TAB} \texttt{CTRL+D})
    \item Редагування командної строки - про яке я казав раніше являє собою просте переназначення функцій клавіш і навішування на них певних команд.
    \item Арифметичне розширення (розкривається чреез '@') дозволяє працювати з змінними що збережені як строки як з числами.
    \item використання RAA (що я вляють собою значення заключені в дужки).
    \end{itemize}

  \subsection{Недоліки}
  Наразі tcsh як і FreeBSD пережеває не найкращі часи: функції що довгий час були фішкою саме tcsh вже давно так чи інакше імплементовано в bash або інші оболонки. Програмування в tcsh не є приємною чи безпечною задачею, перенаправлення потоків працює не зовсім зрозуміло для новачка порівнюю з простотою bash.

  \subsection{Висновки}
  За довгі роки tcsh так і не набула критичної маси користувачів не дивлячись на численні новаторські функціональності і продовжує використовуватись практично лише серед FreeBSD адміністраторів.

\begin{left}\small{\cyr{\textbf{Використані джерела}}}\end{left}
\begin{enumerate}
  \item tcsh (C Shell) Kit Support Guide \\ ftp://ftp.software.ibm.com/s390/zos/unix/ftp/tcsh.pdf
  \item TOPS-20 \\ https://en.wikipedia.org/wiki/TOPS-20#TENEX
  \item М. Собель - Linux. Адміністрування та системне програмування. 2-ге вид. \\
  https://books.google.com.ua/books?id=YfnNlmUFwsMC
  \item Csh Programming Considered Harmful \\
http://www.faqs.org/faqs/unix-faq/shell/csh-whynot/
  \item Командная оболочка tcsh \\ http://citkit.ru/articles/1107/
\end{enumerate}
